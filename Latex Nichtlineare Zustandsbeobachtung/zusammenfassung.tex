\chapter{Zusammenfassung und Ausblick}
\label{ch:zusammenfassung}

Im Rahmen dieser Arbeit wurde eine nichtlineare Zustandsbeobachtung am inversen Doppelpendel durchgef�hrt. Nach einer Modellierung des Doppelpendels wurde als Eingangsexperiment die Zustandsregelung in der instabilen, oberen Gleichgewichtslage simuliert. Die grundlegende Systemeigenschaft der linearen und nichtlinearen Beobachtbarkeit des Doppelpendels wurde untersucht und nachgewiesen. Im Anschluss wurde das EKF als Beobachter f�r die Zustandssch�tzung implementiert. Ein besonderes Augenmerk lag dabei auf den Einstellregeln des EKF. Die Zustandssch�tzungen des EKF wurden mithilfe einer bildbasierten Pendelwinkelmessung validiert. Die Rekonstruktion der Zust�nde des Doppelpendels durch das EKF erweist sich als zufriedenstellend, solange die Nichtlinearit�ten der beobachteten Pendelbewegung nicht zu hoch sind. \\
%
Da das EKF auf einer linearen Approximation des Systems im betrachteten Zustand basiert, liefert es eine zeitweise unkorrekte Zustandssch�tzung bei sehr schnellen Bewegungen des �u�eren Pendelstabs. Auf diese Arbeit aufbauend bietet es sich an, weitere nichtlineare Zustandsbeobachter f�r die Sch�tzung der Pendelwinkel zu verwenden. Ein m�glicher Kandidat ist das Unscented Kalmanfilter (UKF). Dieser Zustandsbeobachter verspricht bessere Ergebnisse hinsichtlich der Zustandssch�tzung nichtlinearer Systeme, vgl. \cite{Wan}. Die bildbasierte Winkelbestimmung mit der \textit{GoPro Hero 4} RGB-Kamera erweist sich als ausreichend f�r die Validierung des Beobachters. Eine Verbesserung der bildbasierten Winkelmessung kann durch eine Kamerakalibrierung bewirkt werden. Mit dem in dieser Arbeit verwendeten Kamerasystem ist die Validierung der Ergebnisse nur \textit{offline} m�glich. Es k�nnte eine {Online}-Validierung implementiert werden. Hierf�r ist ein Kamerasystem, welches eine Schnittstelle zu \textit{Simulink Real-Time} aufweist, notwendig. Au�erdem ist sicherzustellen, dass die Bildverarbeitungsalgorithmen in Echtzeit durchgef�hrt werden k�nnen. Dies kann problematisch sein, da die Verarbeitung der drei Farbkan�le eines Einzelbilds rechenintensiv ist. Es m�sste auch gew�hrleistet werden, dass die Bildverarbeitung bei hohen Bildfrequenzen von �ber \SI{240}{fps} durchf�hrbar ist.  \\
%
Mithilfe der nichtlinearen Zustandssch�tzung k�nnen weitere Experimente am Doppelpendel durchgef�hrt werden. Das EKF rekonstruiert die fehlenden Zust�nde des �u�eren Pendels, wodurch theoretisch die Regelung des Doppelpendels mittels einer Zustandsr�ckf�hrung in die instabile, obere Gleichgewichtslage m�glich ist. Gelingt die Stabilisierung, kann darauf aufbauend die Zustandssch�tzung f�r eine Implementierung eines Aufschwingvorgangs des Doppelpendels aus der stabilen, unteren Gleichgewichtslage verwendet werden.
