\chapter{Stand der Technik}
\label{ch:standdertechnik}

In der Regelungstechnik ist das inverse Doppelpendel ein vielfach untersuchtes System. Dabei l�sst sich die Grundidee der Stabilisierung dieses Systems auf diverse Anwendungsf�lle �bertragen. In~\cite{Olejnik2013} wird die Regelung eines reibungsbehafteten Einrads, welches mithilfe der dynamischen Gleichungen des Doppelpendels modelliert wird, behandelt.  In der humanoiden Robotik ist der aufrechte Stand unter Einwirkung von St�reinfl�ssen eine zu meisternde H�rde. Die Stabilisierung des aufrechten Stands, welche mithilfe des Doppelpendelmodells abstrahiert wird, ist Gegenstand von~\cite{Hettich2014}. Hierbei wird das Verbindungsgelenk der Pendelst�be in die H�fte des menschlichen Modells gesetzt. Die zur Stabilisierung erforderliche Krafteinwirkung erfolgt �ber die F��e. \\\\
%
Die Zustandssch�tzung des �u�eren Pendelwinkels $ \varphi_2 $ wird in~\cite{Samani2010} behandelt. Es wird ein Vergleich zwischen dem Kalmanfilter und der Zustandsbeobachtung durch die \textit{Active Learning Method} (ALM) vollzogen, wobei die ALM bessere Ergebnisse liefert. Neben der Zustandssch�tzung lassen sich Beobachter auch f�r die Identifikation von Systemparametern nutzen. In~\cite{Merwe2004} wird neben einer Zustandssch�tzung der Pendelwinkel, welche f�r die Stabiliserung des Doppelpendels mittels einer optimalen Regelung verwendet wird, eine  Parameteridentifikation der Wagen- und Pendelmassen sowie der Pendell�ngen durchgef�hrt. Die Zustandssch�tzung und Parameteridentifikation basiert auf verrauschten Messungen der Wagenposition, Wagengeschwindigkeit und des �u�eren Pendelwinkels. Der verwendete Beobachter ist das \textit{Unscented Kalmanfilter}. Das Doppelpendel l�sst sich durch Hinzunahme eines dritten Stabs zu einem Dreifachpendel erweitern. In~\cite{Glueck2013} wird der Aufschwingversuch sowie eine Zustandssch�tzung am Dreifachpendel untersucht. Von gro�er Relevanz ist die technische Umsetzung des realen Dreifachpendels, um dieses erfolgreich aufschwingen zu lassen und dann in der oberen Gleichgewichtslage regeln zu k�nnen. \\\\
%
Das besondere an dieser Arbeit ist, dass die Zustandssch�tzung mittels \textit{Erweitertem Kalmanfilter} zun�chst in der Simulation durchgef�hrt und dann am realen Versuchsstand validiert wird. Die Validierung wird mit einer ber�hrungslosen, bildbasierten Winkelbestimmung der Pendelwinkel realisiert. Diese Art der Pendelwinkelbestimmung ver�ndert die Dynamik des Systems nicht.