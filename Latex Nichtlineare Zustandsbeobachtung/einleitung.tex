\chapter{Einleitung}
\label{ch:einleitung}


In der Regelungstechnik gewinnen Zustandsr�ckf�hrungen immer mehr an Bedeutung. Zum einen ist durch die Zustandsr�ckf�hrung eine Auslegung der dynamischen Eigenschaften des Regelkreises m�glich. Instabile Systeme wie das Doppelpendel k�nnen so durch die Manipulation der Eigenwerte des geschlossenen Regelkreises stabilisiert werden. Zum anderen gen�gen bei der Zustandsr�ckf�hrung proportional wirkende Regler, wodurch die Umsetzung des Reglergesetzes durch einfach Additionen und Multiplikation vollzogen werden kann, vgl. \cite{Lunze2010}. F�r die Anwendung einer Zustandsr�ckf�hrung m�ssen jedoch alle Zust�nde vorhanden sein. In vielen F�llen sind nicht alle Systemzust�nde aufgrund fehlender Sensorik oder finanziellem Mehraufwand messbar. Wenn die Voraussetzungen der Beobachtbarkeit eines Systems erf�llt sind, k�nnen Zustandsbeobachter nicht messbare Gr��en rekonstruieren. \\\\
%
Zu Demonstrations- und Forschungszwecken wurde am \textit{Institut f�r Mechatronische Systeme }ein inverses Pendel auf einem verfahrbaren Wagen aufgebaut. Im Rahmen dieser Studienarbeit wird das inverse Pendel mit einem zweiten Stabs zum Doppelpendel erweitert. W�hrend der innere Pendelwinkel gemessen wird, verf�gt das �u�ere Pendel �ber keine Sensorik zur Lagebestimmung. Es ist mithilfe von Zustandsbeobachtern trotzdem m�glich, den �u�eren Pendelwinkel zu sch�tzen. Beobachter rekonstruieren Systemzust�nde unter Verwendung des zugrundeliegenden mathematischen Modells. Die nichtlineare Beobachtung der Zust�nde des inversen Doppelpendels ist Gegenstand dieser Arbeit. \\\\
%
Die Studienarbeit ist wie folgt gegliedert. Zuerst werden die dynamischen Bewegungsgleichungen des inversen Doppelpendels hergeleitet. Als n�chstes wird die Zustandsregelung des Doppelpendels, welche als Motivation f�r die Zustandsbeobachtung dient, in der oberen, stabilen Gleichgewichtslage simuliert. In Kapitel~\ref{ch:zustandsbeobachtung} wird die Systemeigenschaft der Beobachtbarkeit untersucht, um anschlie�end die Implementierung eines nichtlinearen Beobachterkonzepts f�r das inverse Doppelpendel vorzunehmen. Im Rahmen dieser Arbeit wird das Erweiterte Kalmanfilter als Zustandsbeobachter verwendet. Die Winkelsch�tzung der Pendelst�be wird nach einer Erl�uterung der bildbasierten Winkelbestimmung am realen Versuchsstand validiert und die Ergebnisse werden diskutiert. 