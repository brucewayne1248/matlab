\documentclass[]{article}
\usepackage{enumitem}
\usepackage[T1]{fontenc} 
\usepackage{pgfplots}
\listfiles
\setlist[enumerate]{label*=\arabic*.}
\newcounter{saveenumerate}
\makeatletter
\newcommand{\enumeratext}[1]{%
	\setcounter{saveenumerate}{\value{enum\romannumeral\the\@enumdepth}}
\end{enumerate}
#1
\begin{enumerate}
	\setcounter{enum\romannumeral\the\@enumdepth}{\value{saveenumerate}}%
}
\makeatother
%opening
\title{Inhaltsverzeichnis Studienarbeit}
\author{}

\begin{document}

%\maketitle

\hspace{-0.5cm}\textbf{Inhaltsverzeichnis Studienarbeit: Nichtlineare Zustandsbeobachtung mit bildbasierter Validierung am inversen Doppelpendel}
\begin{enumerate}
	\item Einleitung
	\item Stand der Technik/ Motivation
	\item Dynamische Bewegungsgleichungen (Lagrange)
	\item Zustandsregelung (Simulation)
	\begin{enumerate}
		\item Zustandsraumdarstellung (Linearisierung)
		\item Grundlagen Zustandsregelung (Steuerbarkeit, Beobachtbarkeit ...)
		\item Stabilisierung des Doppelpendels mittels LQR (Fazit Zustandsbeobachtung notwendig)
		\item (Stabilisierung des Doppelpendels mittels Polvorgabe, m�glicherweise)
	\end{enumerate}
	\item Zustandsbeobachtung
	\begin{enumerate}
		\item Kalman-Filter Einleitung
		\item Nichtlineare Zustandsbeobachtung mittels EKF
	\end{enumerate}
	\item Bildbasierte Beobachtervalidierung
	\begin{enumerate}
		\item Bildbasierte Winkelmessung
		\item Signalverarbeitung
		\begin{enumerate}
			\item Kreuzkorrelation
			\item Filterung
		\end{enumerate}
		\item Validierung EKF
	\end{enumerate}
	\item Schlussfolgerungen und Ausblick
\end{enumerate}




\end{document}
